\chapter{Introduction to Project}

If software was a person then Silicon would be its water. Si has made the software revolution possible. Semiconductors allow us to realise boolean logics on the nanoscale and ultimately allow us to integrate several of them together to build an Integrated Circuit. ICs play an important role in control applications everywhere. There are ICs everywhere around us, from the air-conditioner cooling our room to the car taking us to office. It is therefore, of importance to us to understand semiconductors, their working and their shortcomings to become good hardware engineers who can develop high quality ICs that not just ease our lives but make scientific and technological innovations easier to achieve. In this report, we start by discussing about Silicon: its fabrication process. We then talk about CMOS fabrication and MOSFET operation. Towards the end we discuss the failure modes of a MOSFET and the short channel effects. Finally, we discuss about NBTI and NBTI testing. Negative Bias Temperature Instability is a failure mode in MOSFETs. NBTI manifests itself in the form of increase in threshold voltage and a consequent decrease of drain current and therefore transconductance of the MOSFET. NBTI is of a concern in pMOSFETs that operate under negative voltage and has been on the rise with incorporation of Nitrogen in the oxide layer and decrease in MOSFET size with every technological node. It is therefore of importance to test fabricated pMOSFETs for NBTI and understanding the test conditions and how to identify a faulty device after testing.